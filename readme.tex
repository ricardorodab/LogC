\documentclass[letterpaper,11pt]{article}


\usepackage[utf8]{inputenc}
\usepackage{amsmath}
\usepackage{amsfonts}
\usepackage{amssymb}
\usepackage{amsthm}
\usepackage[spanish]{babel}
\usepackage{fontenc}
\usepackage{graphicx}

\title{Lógica Computacional 2015-2\\Pŕactica 2}
\author{Jos\'e Ricardo Rodr\'{\i}guez Abreu}
\date{\today\\ Facultad de Ciencias UNAM}

\begin{document}
 
 \maketitle

 
 \begin{center}
 {\bf ¿Qué se me dificultó más en esta práctica?}
 \end{center}
 \\
 En particular la parte mas difícil para mi de la práctica fue implementar la función iForm ya que tuve problemas en ver como se implementaban los predicados. No entendía bien como con el mundo ibamos a agarrar los casos que nos interesaban.

\begin{center} 
 {\bf ¿Qué se me hizo más fácil?}
\end{center}
 \\
 Formalizar los enunciados que vienen en las pruebas. Eso fue relativamente trivial viendo el primer ejemplo e incluso llegando a exagerar, viendo otras prácticas anteriores.
 
\begin{center}
 {\bf Comentarios y sugerencias}
  \end{center}
 \\
 ¡Aprendí algo!
 \end{document}
